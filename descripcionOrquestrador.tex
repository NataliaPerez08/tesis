Este prototipo de sistema permitirá automatizar la configuración de pares, mediante:
\begin{itemize}
    \item \textbf{Cliente}: Programa que se ejecutará en el dispositivo final, el cual consiste de:
    \begin{itemize}
        \item \textit{Cliente daemon}: Un servidor que actuará como daemon guardando, actualizando y manteniendo la configuración actual de los pares y el cliente.
        \item \textit{Interfaz de cliente}: Un programa que actuará como interfaz recibiendo instrucciones por CLI e interactuando con el cliente daemon.
    \end{itemize}
    \item \textbf{Servidor Orquestador (LinkGuard)}: Un servidor que se encargará de orquestar y automatizar la configuración de los pares dentro de las redes privadas.
\end{itemize}


\section{Herramientas a utilizar}
\begin{itemize}
    \item \textbf{Lenguaje de programación}: Python 3.13
    \item \textbf{Framework}: Flask 2.0
    \item \textbf{Sistema de control de versiones}: Git
    \item \textbf{Manejo de dependencias}: Poetry
    \item \textbf{Documentación}: Sphinx
    \item \textbf{Pruebas}: Pytest
\end{itemize}