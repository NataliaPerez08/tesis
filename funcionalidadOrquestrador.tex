\subsection*{Comandos de la aplicación}
Describo a continuación los comandos y sus parámetros: 
    % Crear una red privada
    \begin{Verbatim}[breaklines=true]
    python3 main.py create_private_network <name> <segment> <mask>
    \end{Verbatim}
    Crea una red privada con los siguientes parámetros:
    \begin{itemize}
        \item \texttt{<name>}: Nombre de la red privada.
        \item \texttt{<segment>}: Segmento de red de la red privada.
        \item \texttt{<mask>}: Máscara de red de la red privada.
    \end{itemize}

    %  Crea un red privada solo con el nombre
    \begin{Verbatim}[breaklines=true]
    python3 main.py create_private_network <name>
    \end{Verbatim}
    Crea una red privada con el nombre proporcionado y asigna un segmento y máscara de red por defecto.
    El segmento por defecto es 10.0.0.0 y la máscara por defecto es 28.

    % Listar las redes privadas
    \begin{Verbatim}[breaklines=true]
    python3 main.py get_private_networks
    \end{Verbatim}
    Lista todas las redes privadas disponibles en el sistema. Identifica las redes existentes y sus identificadores únicos.

    % Add a public IP address
    \begin{Verbatim}[breaklines=true]
    python3 main.py add_public_ip <endpoint_id> <ip>
\end{Verbatim}

    Esta función permite agregar una dirección IP pública al sistema. La dirección IP proporcionada en <ip> será registrada por el orquestador para utilizarla en la configuración de conexiones o redes relacionadas. Es útil cuando se requiere registrar una nueva IP que será utilizada por un cliente o un servidor.
    Esta función requiere el identificador único de un peer o endpoint, que se puede obtener mediante el comando \texttt{get\_endpoints}.

    
    % Query a client's public IP address
    \begin{Verbatim}[breaklines=true]
    python3 main.py query_client_public_ip
    \end{Verbatim}

    Esta función consulta y devuelve la dirección IP pública asociada al cliente que ejecuta el comando. Es útil para identificar dinámicamente la IP pública actual del cliente, especialmente en redes con direcciones asignadas dinámicamente por el proveedor de servicios de Internet.
    
    % Registrar un peer en una red privada
    \begin{Verbatim}[breaklines=true]
        python3 main.py register_as_peer <name> <private_network_id> <client_ip> <client_port>
    \end{Verbatim}
        
    Permite registrar un nuevo \textit{peer} en una red privada específica con los siguientes parámetros:  
    \begin{itemize}
        \item \texttt{<name>}: Nombre asignado al \textit{peer}.
        \item \texttt{<private\_network\_id>}: Identificador único de la red privada.
        \item \texttt{<client\_ip>}: Dirección IP del cliente a registrar.
        \item \texttt{<client\_port>}: Puerto del cliente que se usará para las comunicaciones.
    \end{itemize}
        

    % Obtener la clave pública de un cliente
    \begin{Verbatim}[breaklines=true]
    python3 main.py get_client_public_key
    \end{Verbatim} 
    Recupera la clave pública del cliente que ejecuta el comando, necesaria para autenticar las comunicaciones en redes privadas mediante WireGuard.
    
   
    
    % Inicializar una interfaz WireGuard
    \begin{Verbatim}[breaklines=true]
    python3 main.py init_wireguard_interface <client_ip>
    \end{Verbatim} 
    Configura e inicializa una interfaz WireGuard usando la dirección IP proporcionada. Esto permite que el cliente participe en redes privadas gestionadas por WireGuard.
    
   
    
    % Crear un peer en la configuración de WireGuard
    \begin{Verbatim}[breaklines=true]
        python3 main.py create_peer <public_key> <allowed_ips> <client_ip> <listen_port>
    \end{Verbatim}
        
    Agrega un nuevo \textit{peer} a la configuración de WireGuard con los siguientes parámetros:  
        \begin{itemize}
            \item \texttt{<public\_key>}: Clave pública del \textit{peer} a agregar.
            \item \texttt{<allowed\_ips>}: Rango de direcciones IP permitidas para el \textit{peer}.
            \item \texttt{<client\_ip>}: Dirección IP asociada al cliente.
            \item \texttt{<listen\_port>}: Puerto de escucha configurado para el \textit{peer}.
        \end{itemize}
        
   
    
    % Cerrar sesión o terminar una sesión
    \begin{Verbatim}[breaklines=true]
    python3 main.py logout
    \end{Verbatim} 
    Cierra la sesión activa del cliente o usuario. Es útil para garantizar la seguridad del sistema y liberar recursos asociados a la sesión.
    

    % Eliminar una red privada
    \begin{Verbatim}[breaklines=true]
    python3 main.py delete_private_network <private_network_id>
    \end{Verbatim} 
    Elimina una red privada identificada por `<private\_network\_id>`. Este comando elimina todas las configuraciones y peers asociados a la red.
    
    % Eliminar un peer de una red privada
    \begin{Verbatim}[breaklines=true]
    python3 main.py delete_network_peer <private_network_id> <endpoint_id>
    \end{Verbatim} 
    Elimina un peer específico, identificado por `<endpoint\_id>`, de una red privada identificada por `<private\_network\_id>`.
    
    % Editar una red:
    
    % Actualizar el segmento de la red
    \begin{Verbatim}[breaklines=true]
    python3 main.py edit_vpn_segment <private_network_id> <new_segment>
    \end{Verbatim} 
    Modifica el segmento de red de una red privada. Este cambio afecta a todos los hosts conectados a la red y desencadena una actualización en tiempo real.
    
    % Actualizar la máscara de red
    \begin{Verbatim}[breaklines=true]
    python3 main.py edit_vpn_mask <private_network_id> <new_mask>
    \end{Verbatim} 
    Cambia la máscara de red de una red privada. Este cambio también desencadena una actualización en todos los hosts conectados.
    
    % Actualizar el nombre de la red
    \begin{Verbatim}[breaklines=true]
    python3 main.py edit_vpn_name <private_network_id> <new_name>
    \end{Verbatim} 
    Renombra una red privada especificada por `<private\_network\_id>` al nuevo nombre proporcionado en `<new\_name>`.
