Se espera que este prototipo reduzca la superficie de ataque de la red privada virtual al no realizar más funcionalidad que la orquestación de los pares en WireGuard. 


Para evaluar el prototipo se propondrán tres escenarios con dos dispositivos finales:
\begin{itemize}
    \item \textbf{Todos los dispositivos finales son alcanzables}: Es decir, tanto el orquestador como los dispositivos finales pueden comunicarse entre sí porque están en la misma red o cuentan con IPs públicas. No existen restricciones como firewalls o NATs.
    
    \item \textbf{Uno de los dispositivos es alcanzable}: En este escenario, el orquestador y los dispositivos final pueden comunicarse entre sí, pero uno dispositivo final no tiene una IP pública o está detrás de un NAT. De forma que el orquestador actuará como intermediario en la comunicación.
    
    \item \textbf{Solo el Orquestador es alcanzable}: Los dispositivos finales no pueden comunicarse entre sí directamente, pero pueden comunicarse con el orquestador, que actuará como intermediario en la comunicación.
\end{itemize}
