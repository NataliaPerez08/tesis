
El objetivo principal es desarrollar un prototipo open-source de un sistema de control de configuración de pares minimalista usando el protocolo WireGuard, inspirado en Tailscale y adherido a los principios de UNIX de modularidad, claridad, composición, separación, simplicidad, transparencia, solidez, representación, menor sorpresa, silencio y reparación. Este sistema tiene la finalidad de automatizar la configuración y sincronización de pares en una VPN WireGuard.

Permitirá a los usuarios automatizar la configuración de los pares de una VPN WireGuard mediante un servidor orquestador y un cliente que se ejecutará en el dispositivo final. De forma que el cliente en el dispositivo final recibirá una vez la configuración por CLI de los pares y la mantendrá actualizada gracias a que enviará esta información al servidor orquestador que actuará como directorio conociendo la dirección IP del endpoint, puertos, llaves públicas, lista de IPs permitidas por cada par en cada red privada. Además, el servidor orquestador permitirá la comunicación entre dispositivos finales que no pueden comunicarse directamente, ya que actuará como intermediario en la comunicación.

Finalmente en este trabajo se validara la funcionalidad y robustez del prototipo mediante pruebas en diferentes escenarios, evaluando su impacto en la simplificación de la configuración y gestión de redes VPN WireGuard.